% Created 2020-03-23 Mon 19:44
% Intended LaTeX compiler: pdflatex
\input source_header.tex
\author{koo}
\date{\today}
\title{}
\hypersetup{
 pdfauthor={koo},
 pdftitle={},
 pdfkeywords={},
 pdfsubject={},
 pdfcreator={Emacs 26.3 (Org mode 9.2.6)},
 pdflang={English}}

\renewcommand{\docheader}[3]{%

  \thispagestyle{empty}

\markright{SICP, JavaScript Adaptation, #2 #3, #1}
\begin{center}
  {\Large {\bf Specification of #2 #3}---#1 edition}\\[10mm]

  {\large Tee Hao Wei}\\[5mm]

  {\large National University of Singapore \\
          School of Computing}\\[10mm]

  {\large \today}\\[10mm]
\end{center}
}
\begin{document}
	%%%%%%%%%%%%%%%%%%%%%%%%%%%%%%%%%%%%%%%%%%%%%%%
	\docheader{2021}{Source VM Instruction Set}{}
	%%%%%%%%%%%%%%%%%%%%%%%%%%%%%%%%%%%%%%%%%%%%%%%

\input source_intro.tex

\tableofcontents

\label{orgabf1b1f}

\label{org91c3f94}

\section{Source VM Instruction Set}
\label{sec:orgdf5f0e0}
\label{orga35b075}

\label{orgaf84f79}
Table of Contents

\begin{itemize}
\item \hyperref[sec:orgdf5f0e0]{Source VM Instruction Set}

\begin{itemize}
\item \hyperref[sec:orgd013882]{General faults}
\item \hyperref[sec:org6c2c333]{Mnemonic conventions}
\item \hyperref[sec:org6af44d0]{Instruction entry format}

\begin{itemize}
\item \hyperref[sec:org91c15ea]{\texttt{mnemonic}: instruction name}
\end{itemize}

\item \hyperref[sec:org33d7595]{Instructions}

\begin{itemize}
\item \hyperref[sec:orgbc7ec8d]{\texttt{nop}: no-op}
\item \hyperref[sec:orgc756866]{\texttt{ldc.i}: load constant integer}
\item \hyperref[sec:org5cf9345]{\texttt{lgc.i}: load boxed
constant integer}
\item \hyperref[sec:orgd3689f6]{\texttt{ldc.f32}: load
constant number (single-precision)}
\item \hyperref[sec:orgee247e4]{\texttt{lgc.f32}:
load boxed constant number (single-precision)}
\item \hyperref[sec:org6bba761]{\texttt{ldc.f64}: load
constant number (double-precision)}
\item \hyperref[sec:orgae88789]{\texttt{lgc.f64}:
load boxed constant number (double-precision)}
\item \hyperref[sec:org628c82c]{\texttt{ldc.b.0}: load constant false}
\item \hyperref[sec:org45a4edf]{\texttt{ldc.b.1}: load constant true}
\item \hyperref[sec:orgdf93d05]{\texttt{lgc.b.0}: load boxed
constant false}
\item \hyperref[sec:org79cc95f]{\texttt{lgc.b.1}: load boxed
constant true}
\item \hyperref[sec:org5d44e29]{\texttt{lgc.u}: load boxed
constant undefined}
\item \hyperref[sec:org1969f37]{\texttt{lgc.n}: load boxed constant
null}
\item \hyperref[sec:org2f69311]{\texttt{lgc.s}: load constant string}
\item \hyperref[sec:org5cb332a]{\texttt{pop.g}: pop boxed value
from stack}
\item \hyperref[sec:org55a65aa]{\texttt{pop.b}: pop boolean from
stack}
\item \hyperref[sec:orgf652769]{\texttt{pop.f}: pop number from stack}
\item \hyperref[sec:orgd3666a0]{\texttt{add.g}: add boxed values}
\item \hyperref[sec:org117c408]{\texttt{add.f}: add numbers}
\item \hyperref[sec:org379b868]{\texttt{sub.g}: subtract boxed values}
\item \hyperref[sec:org371fd15]{\texttt{sub.f}: subtract numbers}
\item \hyperref[sec:org742c65a]{\texttt{mul.g}: multiply boxed values}
\item \hyperref[sec:org29a8cb0]{\texttt{mul.f}: multiply numbers}
\item \hyperref[sec:org97eb268]{\texttt{div.g}: divide boxed values}
\item \hyperref[sec:org9ea386e]{\texttt{div.f}: divide numbers}
\item \hyperref[sec:org6afbbf6]{\texttt{mod.g}: modulo boxed values}
\item \hyperref[sec:orgf4bf035]{\texttt{mod.f}: modulo numbers}
\item \hyperref[sec:org4404772]{\texttt{not.g}: negate boxed value}
\item \hyperref[sec:org446322c]{\texttt{not.b}: negate boolean}
\item \hyperref[sec:org5479aa3]{\texttt{lt.g}: less than, boxed
operands}
\item \hyperref[sec:orga91f7a6]{\texttt{lt.f}: less than, number
operands}
\item \hyperref[sec:orge89bbc6]{\texttt{gt.g}: greater than, boxed
operands}
\item \hyperref[sec:org3ea3931]{\texttt{gt.f}: greater than,
number operands}
\item \hyperref[sec:orgf9e2550]{\texttt{le.g}: less than
or equal to, boxed operands}
\item \hyperref[sec:orgef83a7a]{\texttt{le.f}: less than
or equal to, number operands}
\item \hyperref[sec:org7e275af]{\texttt{ge.g}: greater
than or equal to, boxed operands}
\item \hyperref[sec:org24757e3]{\texttt{ge.f}: greater
than or equal to, number operands}
\item \hyperref[sec:orgb222de9]{\texttt{eq.g}: equal, boxed operands}
\item \hyperref[sec:orge0a190c]{\texttt{eq.f}: equal, number operands}
\item \hyperref[sec:org73ca0ae]{\texttt{eq.b}: equal, boolean operands}
\item \hyperref[sec:org1d41a06]{\texttt{new.c}: create function}
\item \hyperref[sec:orgcc2c694]{\texttt{new.a}: create array}
\item \hyperref[sec:orgbae78d9]{\texttt{ldl.g}: load
boxed value from current environment}
\item \hyperref[sec:orgb7a98c2]{\texttt{ldl.f}: load
number from current environment}
\item \hyperref[sec:org671ebc7]{\texttt{ldl.b}: load
boolean from current environment}
\item \hyperref[sec:org466d475]{\texttt{stl.g}:
store boxed value into current environment}
\item \hyperref[sec:org711619f]{\texttt{stl.b}: store
boolean into current environment}
\item \hyperref[sec:org29b23b4]{\texttt{stl.f}: store
number into current environment}
\item \hyperref[sec:org445285b]{\texttt{ldp.g}: load
boxed value from (parent) environment}
\item \hyperref[sec:orgf48099d]{\texttt{ldp.f}: load
number from (parent) environment}
\item \hyperref[sec:org74003ff]{\texttt{ldp.b}: load
boolean from (parent) environment}
\item \hyperref[sec:org3861b91]{\texttt{stp.g}:
store boxed value into (parent) environment}
\item \hyperref[sec:org15feb28]{\texttt{stp.b}: store
boolean into (parent) environment}
\item \hyperref[sec:org4696eef]{\texttt{stp.f}: store
number into (parent) environment}
\item \hyperref[sec:org0b3a1ff]{\texttt{lda.g}: load boxed value
from array}
\item \hyperref[sec:orgb1de169]{\texttt{lda.b}: load boolean from
array}
\item \hyperref[sec:orgdc8e9a7]{\texttt{lda.f}: load number from
array}
\item \hyperref[sec:org4724d89]{\texttt{sta.g}: store boxed value
into array}
\item \hyperref[sec:org8e9e3f6]{\texttt{sta.b}: store boolean into
array}
\item \hyperref[sec:org3dfaf54]{\texttt{sta.f}: store number into
array}
\item \hyperref[sec:orge20199f]{\texttt{br.t}: branch if true}
\item \hyperref[sec:org6c72b12]{\texttt{br.f}: branch if false}
\item \hyperref[sec:org71bb9a3]{\texttt{br}: branch}
\item \hyperref[sec:orgeb956da]{\texttt{jmp}: jump}
\item \hyperref[sec:orgc60978b]{\texttt{call}: call function}
\item \hyperref[sec:org026b562]{\texttt{call.t}: tail call function}
\item \hyperref[sec:org99bb90f]{\texttt{call.p}: call primitive
function}
\item \hyperref[sec:orgd22c74d]{\texttt{call.t.p}: tail call
primitive function}
\item \hyperref[sec:org008ae6e]{\texttt{call.v}:
call VM-internal function/native function}
\item \hyperref[sec:org22cfce9]{\texttt{call.t.v}:
tail call VM-internal function/native function}
\item \hyperref[sec:org5c86ae1]{\texttt{ret.g}: return boxed value}
\item \hyperref[sec:org24d7891]{\texttt{ret.f}: return number}
\item \hyperref[sec:org489d37c]{\texttt{ret.b}: return boolean}
\item \hyperref[sec:org4462eb4]{\texttt{ret.u}: return undefined}
\item \hyperref[sec:org02d0940]{\texttt{ret.n}: return null}
\item \hyperref[sec:orgde7ff7a]{\texttt{dup}: duplicate top of stack}
\item \hyperref[sec:orgb7d0128]{\texttt{newenv}: create new
environment}
\item \hyperref[sec:org4cc850b]{\texttt{popenv}: pop environment}
\end{itemize}

\item \hyperref[sec:org22a9c67]{Primitive functions}
\item \hyperref[sec:org5fffbe9]{Machine-parseable instruction
set}
\end{itemize}
\end{itemize}

\subsection{General faults}
\label{sec:orgd013882}
Faults specific to particular instructions are documented in the entries
below.

Aside from those, the following faults can occur; these are presented
here as they apply to multiple instructions.

\begin{itemize}
\item Pushing onto a full stack

\item Popping from an empty stack

\item Loading from or storing into an invalid index in an environment

\item Running out of memory
\end{itemize}

\subsection{Mnemonic conventions}
\label{sec:org6c2c333}
For instructions that operate on values, the type of the operand(s) is
represented in the instruction mnemonic by a letter:

\begin{itemize}
\item \texttt{u} undefined

\item \texttt{n} null

\item \texttt{b} boolean

\item \texttt{i} integer

\item \texttt{f} number (float)

\item \texttt{s} string

\item \texttt{a} array

\item \texttt{c} function

\item \texttt{g} boxed value
\end{itemize}

\subsection{Instruction entry format}
\label{sec:org6af44d0}
\subsubsection{\texttt{mnemonic}: instruction name}
\label{sec:org91c15ea}
Format: \texttt{opcode <argument1: type> <argument2: type>} (N bytes)

The binary format byte count is omitted for instructions with no
arguments. They are 1 byte long.

Stack before: \ldots{}, \texttt{value1: type}, \texttt{value2: type} \(\blacksquare\)

Stack after: \ldots{}, \texttt{value3: type} \(\blacksquare\)

\(\blacksquare\) represents the top of the stack.

The rest of the entry details what the opcode does, and any possible
faults.

\subsection{Instructions}
\label{sec:org33d7595}
\subsubsection{\texttt{nop}: no-op}
\label{sec:orgbc7ec8d}
Format: \texttt{0x00}

Does not modify the stack.

Does nothing.

\subsubsection{\texttt{ldc.i}: load constant integer}
\label{sec:orgc756866}
Format: \texttt{0x01 <value: i32>} (5 bytes)

Stack before: \ldots{} \(\blacksquare\)

Stack after: \ldots{}, \texttt{<value>: number} \(\blacksquare\)

Pushes an integer with value equal to \texttt{<value>} onto the stack.

\subsubsection{\texttt{lgc.i}: load boxed constant integer}
\label{sec:org5cf9345}
Format: \texttt{0x02 <value: i32>} (5 bytes)

Stack before: \ldots{} \(\blacksquare\)

Stack after: \ldots{}, \texttt{<value>: boxed (number)} \(\blacksquare\)

Pushes a boxed integer with value equal to \texttt{<value>} onto the stack.

\subsubsection{\texttt{ldc.f32}: load constant number (single-precision)}
\label{sec:orgd3689f6}
Format: \texttt{0x03 <number: f32>} (5 bytes)

Stack before: \ldots{} \(\blacksquare\)

Stack after: \ldots{}, \texttt{<number>: number} \(\blacksquare\)

Pushes a number with value equal to \texttt{<number>} onto the stack.

\subsubsection{\texttt{lgc.f32}: load boxed constant number (single-precision)}
\label{sec:orgee247e4}
Format: \texttt{0x04 <number: f32>} (5 bytes)

Stack before: \ldots{} \(\blacksquare\)

Stack after: \ldots{}, \texttt{<number>: boxed (number)} \(\blacksquare\)

Pushes a boxed number with value equal to \texttt{<number>} onto the stack.

\subsubsection{\texttt{ldc.f64}: load constant number (double-precision)}
\label{sec:org6bba761}
Format: \texttt{0x05 <number: f64>} (9 bytes)

Stack before: \ldots{} \(\blacksquare\)

Stack after: \ldots{}, \texttt{<number>: number} \(\blacksquare\)

Pushes a number with value equal to \texttt{<number>} onto the stack.

\subsubsection{\texttt{lgc.f64}: load boxed constant number (double-precision)}
\label{sec:orgae88789}
Format: \texttt{0x06 <number: f64>} (9 bytes)

Stack before: \ldots{} \(\blacksquare\)

Stack after: \ldots{}, \texttt{<number>: boxed (number)} \(\blacksquare\)

Pushes a boxed number with value equal to \texttt{<number>} onto the stack.

\subsubsection{\texttt{ldc.b.0}: load constant false}
\label{sec:org628c82c}
Format: \texttt{0x07}

Stack before: \ldots{} \(\blacksquare\)

Stack after: \ldots{}, \texttt{false: boolean} \(\blacksquare\)

Pushes the boolean \texttt{false} onto the stack.

\subsubsection{\texttt{ldc.b.1}: load constant true}
\label{sec:org45a4edf}
Format: \texttt{0x08}

Stack before: \ldots{} \(\blacksquare\)

Stack after: \ldots{}, \texttt{true: boolean} \(\blacksquare\)

Pushes the boolean \texttt{true} onto the stack.

\subsubsection{\texttt{lgc.b.0}: load boxed constant false}
\label{sec:orgdf93d05}
Format: \texttt{0x09}

Stack before: \ldots{} \(\blacksquare\)

Stack after: \ldots{}, \texttt{false: boxed (boolean)} \(\blacksquare\)

Pushes a boxed boolean \texttt{false} onto the stack.

\subsubsection{\texttt{lgc.b.1}: load boxed constant true}
\label{sec:org79cc95f}
Format: \texttt{0x0A}

Stack before: \ldots{} \(\blacksquare\)

Stack after: \ldots{}, \texttt{true: boxed (boolean)} \(\blacksquare\)

Pushes a boxed boolean \texttt{true} onto the stack.

\subsubsection{\texttt{lgc.u}: load boxed constant undefined}
\label{sec:org5d44e29}
Format: \texttt{0x0B}

Stack before: \ldots{} \(\blacksquare\)

Stack after: \ldots{}, \texttt{undefined: boxed (undefined)} \(\blacksquare\)

Pushes a boxed \texttt{undefined} onto the stack.

\subsubsection{\texttt{lgc.n}: load boxed constant null}
\label{sec:org1969f37}
Format: \texttt{0x0C}

Stack before: \ldots{} \(\blacksquare\)

Stack after: \ldots{}, \texttt{null: boxed (null)} \(\blacksquare\)

Pushes a boxed \texttt{null} onto the stack.

\subsubsection{\texttt{lgc.s}: load constant string}
\label{sec:org2f69311}
Format: \texttt{0x0D <address>} (5 bytes)

In the JSON format, the string literal is specified directly instead.

Stack before: \ldots{} \(\blacksquare\)

Stack after: \ldots{}, \texttt{<string>: boxed (string)} \(\blacksquare\)

Pushes the string at the given address onto the stack.

The string at the given address should be null-terminated.

\subsubsection{\texttt{pop.g}: pop boxed value from stack}
\label{sec:org5cb332a}
Format: \texttt{0x0E}

Stack before: \ldots{}, \texttt{<value>: boxed} \(\blacksquare\)

Stack after: \ldots{} \(\blacksquare\)

Pops a boxed value off the stack.

Behaviour is undefined if \texttt{<value>} is not a boxed value.

\subsubsection{\texttt{pop.b}: pop boolean from stack}
\label{sec:org55a65aa}
Format: \texttt{0x0F}

Stack before: \ldots{}, \texttt{<value>: boolean} \(\blacksquare\)

Stack after: \ldots{} \(\blacksquare\)

Pops a boolean off the stack.

Behaviour is undefined if \texttt{<value>} is not a boolean.

\subsubsection{\texttt{pop.f}: pop number from stack}
\label{sec:orgf652769}
Format: \texttt{0x10}

Stack before: \ldots{}, \texttt{<value>: number} \(\blacksquare\)

Stack after: \ldots{} \(\blacksquare\)

Pops a number off the stack.

Behaviour is undefined if \texttt{<value>} is not a number.

\subsubsection{\texttt{add.g}: add boxed values}
\label{sec:orgd3666a0}
Format: \texttt{0x11}

Stack before: \ldots{}, \texttt{<a>: boxed}, \texttt{<b>: boxed} \(\blacksquare\)

Stack after: \ldots{}, \texttt{<c>: boxed} \(\blacksquare\)

Pops \texttt{<a>} and \texttt{<b>} off the stack.

If \texttt{<a>} and \texttt{<b>} are not boxed values, behaviour is undefined.

If \texttt{<a>} and \texttt{<b>} are strings, pushes their concatenation
\texttt{<c> = <a> + <b>} onto the stack.

If \texttt{<a>} and \texttt{<b>} are numbers, pushes their sum \texttt{<c> = <a> + <b>} onto
the stack.

Otherwise, a fault occurs.

\subsubsection{\texttt{add.f}: add numbers}
\label{sec:org117c408}
Format: \texttt{0x12}

Stack before: \ldots{}, \texttt{<a>: number}, \texttt{<b>: number} \(\blacksquare\)

Stack after: \ldots{}, \texttt{<c>: number} \(\blacksquare\)

Pops \texttt{<a>} and \texttt{<b>} off the stack, then pushes their sum
\texttt{<c> = <a> + <b>} onto the stack.

If \texttt{<a>} and \texttt{<b>} are not numbers, behaviour is undefined.

\subsubsection{\texttt{sub.g}: subtract boxed values}
\label{sec:org379b868}
Format: \texttt{0x13}

Stack before: \ldots{}, \texttt{<a>: boxed}, \texttt{<b>: boxed} \(\blacksquare\)

Stack after: \ldots{}, \texttt{<c>: boxed} \(\blacksquare\)

Pops \texttt{<a>} and \texttt{<b>} off the stack.

If \texttt{<a>} and \texttt{<b>} are not boxed values, behaviour is undefined.

If \texttt{<a>} and \texttt{<b>} are numbers, pushes their difference
\texttt{<c> = <a> - <b>} onto the stack.

Otherwise, a fault occurs.

\subsubsection{\texttt{sub.f}: subtract numbers}
\label{sec:org371fd15}
Format: \texttt{0x14}

Stack before: \ldots{}, \texttt{<a>: number}, \texttt{<b>: number} \(\blacksquare\)

Stack after: \ldots{}, \texttt{<c>: number} \(\blacksquare\)

Pops \texttt{<a>} and \texttt{<b>} off the stack, then pushes their difference
\texttt{<c> = <a> - <b>} onto the stack.

If \texttt{<a>} and \texttt{<b>} are not numbers, behaviour is undefined.

\subsubsection{\texttt{mul.g}: multiply boxed values}
\label{sec:org742c65a}
Format: \texttt{0x15}

Stack before: \ldots{}, \texttt{<a>: boxed}, \texttt{<b>: boxed} \(\blacksquare\)

Stack after: \ldots{}, \texttt{<c>: boxed} \(\blacksquare\)

Pops \texttt{<a>} and \texttt{<b>} off the stack.

If \texttt{<a>} and \texttt{<b>} are not boxed values, behaviour is undefined.

If \texttt{<a>} and \texttt{<b>} are numbers, pushes their product \texttt{<c> = <a> * <b>}
onto the stack.

Otherwise, a fault occurs.

\subsubsection{\texttt{mul.f}: multiply numbers}
\label{sec:org29a8cb0}
Format: \texttt{0x16}

Stack before: \ldots{}, \texttt{<a>: number}, \texttt{<b>: number} \(\blacksquare\)

Stack after: \ldots{}, \texttt{<c>: number} \(\blacksquare\)

Pops \texttt{<a>} and \texttt{<b>} off the stack, then pushes their product
\texttt{<c> = <a> * <b>} onto the stack.

If \texttt{<b>} is the number 0, a fault occurs.

If \texttt{<a>} and \texttt{<b>} are not numbers, behaviour is undefined.

\subsubsection{\texttt{div.g}: divide boxed values}
\label{sec:org97eb268}
Format: \texttt{0x17}

Stack before: \ldots{}, \texttt{<a>: boxed}, \texttt{<b>: boxed} \(\blacksquare\)

Stack after: \ldots{}, \texttt{<c>: boxed} \(\blacksquare\)

Pops \texttt{<a>} and \texttt{<b>} off the stack.

If \texttt{<a>} and \texttt{<b>} are not boxed values, behaviour is undefined.

If \texttt{<a>} and \texttt{<b>} are numbers, pushes their quotient \texttt{<c> = <a> / <b>}
onto the stack.

Otherwise, a fault occurs.

\subsubsection{\texttt{div.f}: divide numbers}
\label{sec:org9ea386e}
Format: \texttt{0x18}

Stack before: \ldots{}, \texttt{<a>: number}, \texttt{<b>: number} \(\blacksquare\)

Stack after: \ldots{}, \texttt{<c>: number} \(\blacksquare\)

Pops \texttt{<a>} and \texttt{<b>} off the stack, then pushes their quotient
\texttt{<c> = <a> / <b>} onto the stack.

If \texttt{<a>} and \texttt{<b>} are not numbers, behaviour is undefined.

\subsubsection{\texttt{mod.g}: modulo boxed values}
\label{sec:org6afbbf6}
Format: \texttt{0x19}

Stack before: \ldots{}, \texttt{<a>: boxed}, \texttt{<b>: boxed} \(\blacksquare\)

Stack after: \ldots{}, \texttt{<c>: boxed} \(\blacksquare\)

Pops \texttt{<a>} and \texttt{<b>} off the stack.

If \texttt{<a>} and \texttt{<b>} are not boxed values, behaviour is undefined.

If \texttt{<a>} and \texttt{<b>} are numbers, pushes \texttt{<c> = <a> \% <b>} onto the stack.

Otherwise, a fault occurs.

\subsubsection{\texttt{mod.f}: modulo numbers}
\label{sec:orgf4bf035}
Format: \texttt{0x1A}

Stack before: \ldots{}, \texttt{<a>: number}, \texttt{<b>: number} \(\blacksquare\)

Stack after: \ldots{}, \texttt{<c>: number} \(\blacksquare\)

Pops \texttt{<a>} and \texttt{<b>} off the stack, then pushes \texttt{<c> = <a> \% <b>} onto
the stack.

If \texttt{<a>} and \texttt{<b>} are not numbers, behaviour is undefined.

\subsubsection{\texttt{not.g}: negate boxed value}
\label{sec:org4404772}
Format: \texttt{0x1B}

Stack before: \ldots{}, \texttt{<a>: boxed} \(\blacksquare\)

Stack after: \ldots{}, \texttt{<b>: boxed} \(\blacksquare\)

Pops \texttt{<a>} off the stack.

If \texttt{<a>} is a boolean, pushes its negation \texttt{<b>} onto the stack.

Otherwise, a fault occurs.

\subsubsection{\texttt{not.b}: negate boolean}
\label{sec:org446322c}
Format: \texttt{0x1C}

Stack before: \ldots{}, \texttt{<a>: boolean} \(\blacksquare\)

Stack after: \ldots{}, \texttt{<b>: boolean} \(\blacksquare\)

Pops \texttt{<a>} off the stack, then pushes its negation \texttt{<b>} onto the stack.

If \texttt{<a>} is not a boolean, behaviour is undefined.

\subsubsection{\texttt{lt.g}: less than, boxed operands}
\label{sec:org5479aa3}
Format: \texttt{0x1D}

Stack before: \ldots{}, \texttt{<a>: boxed}, \texttt{<b>: boxed} \(\blacksquare\)

Stack after: \ldots{}, \texttt{<c>: boolean} \(\blacksquare\)

Pops \texttt{<a>} and \texttt{<b>} off the stack.

If \texttt{<a>} and \texttt{<b>} are not boxed values, behaviour is undefined.

If \texttt{<a>} and \texttt{<b>} are both strings or both numbers, pushes
\texttt{<c> = <a> < <b>}, with the obvious meaning if both operands are
numbers, and comparing the operands by lexicographical order if they are
strings.

Otherwise, a fault occurs.

\subsubsection{\texttt{lt.f}: less than, number operands}
\label{sec:orga91f7a6}
Format: \texttt{0x1E}

Stack before: \ldots{}, \texttt{<a>: number}, \texttt{<b>: number} \(\blacksquare\)

Stack after: \ldots{}, \texttt{<c>: boolean} \(\blacksquare\)

Pops \texttt{<a>} and \texttt{<b>} off the stack, then pushes \texttt{<c> = <a> < <b>} onto
the stack.

If \texttt{<a>} and \texttt{<b>} are not numbers, behaviour is undefined.

\subsubsection{\texttt{gt.g}: greater than, boxed operands}
\label{sec:orge89bbc6}
Format: \texttt{0x1F}

Stack before: \ldots{}, \texttt{<a>: boxed}, \texttt{<b>: boxed} \(\blacksquare\)

Stack after: \ldots{}, \texttt{<c>: boolean} \(\blacksquare\)

Pops \texttt{<a>} and \texttt{<b>} off the stack.

If \texttt{<a>} and \texttt{<b>} are not boxed values, behaviour is undefined.

If \texttt{<a>} and \texttt{<b>} are both strings or both numbers, pushes
\texttt{<c> = <a> > <b>}, with the obvious meaning if both operands are
numbers, and comparing the operands by lexicographical order if they are
strings.

Otherwise, a fault occurs.

\subsubsection{\texttt{gt.f}: greater than, number operands}
\label{sec:org3ea3931}
Format: \texttt{0x20}

Stack before: \ldots{}, \texttt{<a>: number}, \texttt{<b>: number} \(\blacksquare\)

Stack after: \ldots{}, \texttt{<c>: boolean} \(\blacksquare\)

Pops \texttt{<a>} and \texttt{<b>} off the stack, then pushes \texttt{<c> = <a> > <b>} onto
the stack.

If \texttt{<a>} and \texttt{<b>} are not numbers, behaviour is undefined.

\subsubsection{\texttt{le.g}: less than or equal to, boxed operands}
\label{sec:orgf9e2550}
Format: \texttt{0x21}

Stack before: \ldots{}, \texttt{<a>: boxed}, \texttt{<b>: boxed} \(\blacksquare\)

Stack after: \ldots{}, \texttt{<c>: boolean} \(\blacksquare\)

Pops \texttt{<a>} and \texttt{<b>} off the stack.

If \texttt{<a>} and \texttt{<b>} are not boxed values, behaviour is undefined.

If \texttt{<a>} and \texttt{<b>} are both strings or both numbers, pushes
\texttt{<c> = <a> <} <b>=, with the obvious meaning if both operands are
numbers, and comparing the operands by lexicographical order if they are
strings.

Otherwise, a fault occurs.

\subsubsection{\texttt{le.f}: less than or equal to, number operands}
\label{sec:orgef83a7a}
Format: \texttt{0x22}

Stack before: \ldots{}, \texttt{<a>: number}, \texttt{<b>: number} \(\blacksquare\)

Stack after: \ldots{}, \texttt{<c>: boolean} \(\blacksquare\)

Pops \texttt{<a>} and \texttt{<b>} off the stack, then pushes \texttt{<c> = <a> <} <b>= onto
the stack.

If \texttt{<a>} and \texttt{<b>} are not numbers, behaviour is undefined.

\subsubsection{\texttt{ge.g}: greater than or equal to, boxed operands}
\label{sec:org7e275af}
Format: \texttt{0x23}

Stack before: \ldots{}, \texttt{<a>: boxed}, \texttt{<b>: boxed} \(\blacksquare\)

Stack after: \ldots{}, \texttt{<c>: boolean} \(\blacksquare\)

Pops \texttt{<a>} and \texttt{<b>} off the stack.

If \texttt{<a>} and \texttt{<b>} are not boxed values, behaviour is undefined.

If \texttt{<a>} and \texttt{<b>} are both strings or both numbers, pushes
\texttt{<c> = <a> >} <b>=, with the obvious meaning if both operands are
numbers, and comparing the operands by lexicographical order if they are
strings.

Otherwise, a fault occurs.

\subsubsection{\texttt{ge.f}: greater than or equal to, number operands}
\label{sec:org24757e3}
Format: \texttt{0x24}

Stack before: \ldots{}, \texttt{<a>: number}, \texttt{<b>: number} \(\blacksquare\)

Stack after: \ldots{}, \texttt{<c>: boolean} \(\blacksquare\)

Pops \texttt{<a>} and \texttt{<b>} off the stack, then pushes \texttt{<c> = <a> >} <b>= onto
the stack.

If \texttt{<a>} and \texttt{<b>} are not numbers, behaviour is undefined.

\subsubsection{\texttt{eq.g}: equal, boxed operands}
\label{sec:orgb222de9}
Format: \texttt{0x25}

Stack before: \ldots{}, \texttt{<a>: boxed}, \texttt{<b>: boxed} \(\blacksquare\)

Stack after: \ldots{}, \texttt{<c>: boolean} \(\blacksquare\)

Pops \texttt{<a>} and \texttt{<b>} off the stack.

If \texttt{<a>} and \texttt{<b>} are not boxed values, behaviour is undefined.

If \texttt{<a>} and \texttt{<b>} are of different types, pushes \texttt{false} onto the
stack.

If \texttt{<a>} and \texttt{<b>} are both \texttt{undefined} or both \texttt{null}, pushes \texttt{true}
onto the stack.

If \texttt{<a>} and \texttt{<b>} are both booleans, both numbers or both strings,
pushes \texttt{true} onto the stack if they have the same value, otherwise
pushes \texttt{false} onto the stack.

If \texttt{<a>} and \texttt{<b>} are both functions or both arrays, pushes \texttt{true} onto
the stack if they are referentially equal i.e. they refer to the exact
same array or function object, otherwise pushes \texttt{false} onto the stack.

The above cases are exhaustive.

\subsubsection{\texttt{eq.f}: equal, number operands}
\label{sec:orge0a190c}
Format: \texttt{0x26}

Stack before: \ldots{}, \texttt{<a>: number}, \texttt{<b>: number} \(\blacksquare\)

Stack after: \ldots{}, \texttt{<c>: boolean} \(\blacksquare\)

Pops \texttt{<a>} and \texttt{<b>} off the stack, then pushes \texttt{<c> = <a> =} <b>= onto
the stack.

If \texttt{<a>} and \texttt{<b>} are not numbers, behaviour is undefined.

\subsubsection{\texttt{eq.b}: equal, boolean operands}
\label{sec:org73ca0ae}
Format: \texttt{0x27}

Stack before: \ldots{}, \texttt{<a>: boolean}, \texttt{<b>: boolean} \(\blacksquare\)

Stack after: \ldots{}, \texttt{<c>: boolean} \(\blacksquare\)

Pops \texttt{<a>} and \texttt{<b>} off the stack, then pushes \texttt{<c> = <a> =} <b>= onto
the stack.

If \texttt{<a>} and \texttt{<b>} are not booleans, behaviour is undefined.

\subsubsection{\texttt{new.c}: create function}
\label{sec:org1d41a06}
Format: \texttt{0x28 <address>} (5 bytes)

Stack before: \ldots{} \(\blacksquare\)

Stack after: \ldots{}, \texttt{<c>: boxed (function)} \(\blacksquare\)

Pushes a new function object \texttt{<c>} onto the stack referring to the
function at the given address.

\subsubsection{\texttt{new.a}: create array}
\label{sec:orgcc2c694}
Format: \texttt{0x29}

Stack before: \ldots{} \(\blacksquare\)

Stack after: \ldots{}, \texttt{<a>: boxed (array)} \(\blacksquare\)

Pushes a new empty array onto the stack.

\subsubsection{\texttt{ldl.g}: load boxed value from current environment}
\label{sec:orgbae78d9}
Format: \texttt{0x2A <index: u8>} (2 bytes)

Stack before: \ldots{} \(\blacksquare\)

Stack after: \ldots{}, \texttt{<a>: boxed} \(\blacksquare\)

Pushes the value \texttt{<a>} at index \texttt{index} in the current environment onto
the stack.

If the value at index \texttt{index} in the current environment is not a boxed
value, behaviour is undefined.

\subsubsection{\texttt{ldl.f}: load number from current environment}
\label{sec:orgb7a98c2}
Format: \texttt{0x2B <index: u8>} (2 bytes)

Stack before: \ldots{} \(\blacksquare\)

Stack after: \ldots{}, \texttt{<a>: number} \(\blacksquare\)

Pushes the value \texttt{<a>} at index \texttt{index} in the current environment onto
the stack.

If the value at index \texttt{index} in the current environment is not a
number, behaviour is undefined.

\subsubsection{\texttt{ldl.b}: load boolean from current environment}
\label{sec:org671ebc7}
Format: \texttt{0x2C <index: u8>} (2 bytes)

Stack before: \ldots{} \(\blacksquare\)

Stack after: \ldots{}, \texttt{<a>: boolean} \(\blacksquare\)

Pushes the value \texttt{<a>} at index \texttt{index} in the current environment onto
the stack.

If the value at index \texttt{index} in the current environment is not a
boolean, behaviour is undefined.

\subsubsection{\texttt{stl.g}: store boxed value into current environment}
\label{sec:org466d475}
Format: \texttt{0x2D <index: u8>} (2 bytes)

Stack before: \ldots{}, \texttt{<a>: boxed} \(\blacksquare\)

Stack after: \ldots{} \(\blacksquare\)

Pops \texttt{<a>} off the stack and stores it in index \texttt{index} in the current
environment.

If \texttt{<a>} is not a boxed value, behaviour is undefined.

\subsubsection{\texttt{stl.b}: store boolean into current environment}
\label{sec:org711619f}
Format: \texttt{0x2E <index: u8>} (2 bytes)

Stack before: \ldots{}, \texttt{<a>: boolean} \(\blacksquare\)

Stack after: \ldots{} \(\blacksquare\)

Pops \texttt{<a>} off the stack and stores it in index \texttt{index} in the current
environment.

If \texttt{<a>} is not a boolean, behaviour is undefined.

\subsubsection{\texttt{stl.f}: store number into current environment}
\label{sec:org29b23b4}
Format: \texttt{0x2F <index: u8>} (2 bytes)

Stack before: \ldots{}, \texttt{<a>: number} \(\blacksquare\)

Stack after: \ldots{} \(\blacksquare\)

Pops \texttt{<a>} off the stack and stores it in index \texttt{index} in the current
environment.

If \texttt{<a>} is not a number, behaviour is undefined.

\subsubsection{\texttt{ldp.g}: load boxed value from (parent) environment}
\label{sec:org445285b}
Format: \texttt{0x30 <index: u8> <envindex: u8>} (3 bytes)

Stack before: \ldots{} \(\blacksquare\)

Stack after: \ldots{}, \texttt{<a>: boxed} \(\blacksquare\)

Pushes the value at index \texttt{index} in the \texttt{envindex=th parent of the
current environment onto the stack. If =envindex} is \texttt{0}, this is
equivalent to \texttt{ldl.g}.

If the value is not a boxed value, behaviour is undefined.

\subsubsection{\texttt{ldp.f}: load number from (parent) environment}
\label{sec:orgf48099d}
Format: \texttt{0x31 <index: u8> <envindex: u8>} (3 bytes)

Stack before: \ldots{} \(\blacksquare\)

Stack after: \ldots{}, \texttt{<a>: number} \(\blacksquare\)

Pushes the value at index \texttt{index} in the \texttt{envindex=th parent of the
current environment onto the stack. If =envindex} is \texttt{0}, this is
equivalent to \texttt{ldl.f}.

If the value is not a number, behaviour is undefined.

\subsubsection{\texttt{ldp.b}: load boolean from (parent) environment}
\label{sec:org74003ff}
Format: \texttt{0x32 <index: u8> <envindex: u8>} (3 bytes)

Stack before: \ldots{} \(\blacksquare\)

Stack after: \ldots{}, \texttt{<a>: boolean} \(\blacksquare\)

Pushes the value at index \texttt{index} in the \texttt{envindex=th parent of the
current environment onto the stack. If =envindex} is \texttt{0}, this is
equivalent to \texttt{ldl.f}.

If the value is not a boolean, behaviour is undefined.

\subsubsection{\texttt{stp.g}: store boxed value into (parent) environment}
\label{sec:org3861b91}
Format: \texttt{0x33 <index: u8> <envindex: u8>} (3 bytes)

Stack before: \ldots{}, \texttt{<a>: boxed} \(\blacksquare\)

Stack after: \ldots{} \(\blacksquare\)

Pops \texttt{<a>} off the stack, and stores \texttt{<a>} into index \texttt{index} in the
=envindex=th parent of the current environment.

If \texttt{<a>} is not a boxed value, behaviour is undefined.

\subsubsection{\texttt{stp.b}: store boolean into (parent) environment}
\label{sec:org15feb28}
Format: \texttt{0x34 <index: u8> <envindex: u8>} (3 bytes)

Stack before: \ldots{}, \texttt{<a>: boolean} \(\blacksquare\)

Stack after: \ldots{} \(\blacksquare\)

Pops \texttt{<a>} off the stack, and stores \texttt{<a>} into index \texttt{index} in the
=envindex=th parent of the current environment.

If \texttt{<a>} is not a boolean, behaviour is undefined.

\subsubsection{\texttt{stp.f}: store number into (parent) environment}
\label{sec:org4696eef}
Format: \texttt{0x35 <index: u8> <envindex: u8>} (3 bytes)

Stack before: \ldots{}, \texttt{<a>: number} \(\blacksquare\)

Stack after: \ldots{} \(\blacksquare\)

Pops \texttt{<a>} off the stack, and stores \texttt{<a>} into index \texttt{index} in the
=envindex=th parent of the current environment.

If \texttt{<a>} is not a number, behaviour is undefined.

\subsubsection{\texttt{lda.g}: load boxed value from array}
\label{sec:org0b3a1ff}
Format: \texttt{0x36}

Stack before: \ldots{}, \texttt{<array>: boxed (array)}, \texttt{<index>: number} \(\blacksquare\)

Stack after: \ldots{}, \texttt{<value>: boxed} \(\blacksquare\)

Pops \texttt{index} and \texttt{array} off the stack.

If \texttt{index} is a non-negative integer, pushes the value at index \texttt{index}
in \texttt{array} onto the stack.

Otherwise, a fault occurs.

Behaviour is undefined if the value at index \texttt{index} in array is not a
boxed value.

\subsubsection{\texttt{lda.b}: load boolean from array}
\label{sec:orgb1de169}
Format: \texttt{0x37}

Stack before: \ldots{}, \texttt{<array>: boxed (array)}, \texttt{<index>: number} \(\blacksquare\)

Stack after: \ldots{}, \texttt{<value>: boolean} \(\blacksquare\)

Pops \texttt{index} and \texttt{array} off the stack.

If \texttt{index} is a non-negative integer, pushes the value at index \texttt{index}
in \texttt{array} onto the stack.

Otherwise, a fault occurs.

Behaviour is undefined if the value at index \texttt{index} in array is not a
boolean.

\subsubsection{\texttt{lda.f}: load number from array}
\label{sec:orgdc8e9a7}
Format: \texttt{0x38}

Stack before: \ldots{}, \texttt{<array>: boxed (array)}, \texttt{<index>: number} \(\blacksquare\)

Stack after: \ldots{}, \texttt{<value>: number} \(\blacksquare\)

Pops \texttt{index} and \texttt{array} off the stack.

If \texttt{index} is a non-negative integer, pushes the value at index \texttt{index}
in \texttt{array} onto the stack.

Otherwise, a fault occurs.

Behaviour is undefined if the value at index \texttt{index} in array is not a
number.

\subsubsection{\texttt{sta.g}: store boxed value into array}
\label{sec:org4724d89}
Format: \texttt{0x39}

Stack before: \ldots{}, \texttt{<array>: boxed (array)}, \texttt{<index>: number},
\texttt{<value>: boxed} \(\blacksquare\)

Stack after: \ldots{} \(\blacksquare\)

Pops \texttt{value}, \texttt{index} and \texttt{array} off the stack.

If \texttt{index} is a non-negative integer, stores \texttt{value} into index \texttt{index}
in \texttt{array}.

Otherwise, a fault occurs.

Behaviour is undefined if \texttt{value} is not a boxed value.

\subsubsection{\texttt{sta.b}: store boolean into array}
\label{sec:org8e9e3f6}
Format: \texttt{0x3A}

Stack before: \ldots{}, \texttt{<array>: boxed (array)}, \texttt{<index>: number},
\texttt{<value>: boolean} \(\blacksquare\)

Stack after: \ldots{} \(\blacksquare\)

Pops \texttt{value}, \texttt{index} and \texttt{array} off the stack.

If \texttt{index} is a non-negative integer, stores \texttt{value} into index \texttt{index}
in \texttt{array}.

Otherwise, a fault occurs.

Behaviour is undefined if \texttt{value} is not a boolean.

\subsubsection{\texttt{sta.f}: store number into array}
\label{sec:org3dfaf54}
Format: \texttt{0x3B}

Stack before: \ldots{}, \texttt{<array>: boxed (array)}, \texttt{<index>: number},
\texttt{<value>: number} \(\blacksquare\)

Stack after: \ldots{} \(\blacksquare\)

Pops \texttt{value}, \texttt{index} and \texttt{array} off the stack.

If \texttt{index} is a non-negative integer, stores \texttt{value} into index \texttt{index}
in \texttt{array}.

Otherwise, a fault occurs.

Behaviour is undefined if \texttt{value} is not a number.

\subsubsection{\texttt{br.t}: branch if true}
\label{sec:orge20199f}
Format: \texttt{0x3C <offset>} (5 bytes)

Stack before: \ldots{}, \texttt{<condition>: boolean} \(\blacksquare\)

Stack after: \ldots{} \(\blacksquare\)

Pops \texttt{condition} off the stack.

If \texttt{condition} is a boolean and is true, skips \texttt{offset} bytes starting
from after the current instruction. That is, \texttt{br.t 0} is a no-op.

Note that \texttt{offset} can be negative.

If \texttt{condition} is a boolean and is false, does nothing.

Otherwise, a fault occurs.

\subsubsection{\texttt{br.f}: branch if false}
\label{sec:org6c72b12}
Format: \texttt{0x3D <offset>} (5 bytes)

Stack before: \ldots{}, \texttt{<condition>: boolean} \(\blacksquare\)

Stack after: \ldots{} \(\blacksquare\)

Pops \texttt{condition} off the stack.

If \texttt{condition} is a boolean and is false, skips \texttt{offset} bytes starting
from after the current instruction. That is, \texttt{br.f 0} is a no-op.

Note that \texttt{offset} can be negative.

If \texttt{condition} is a boolean and is true, does nothing.

Otherwise, a fault occurs.

\subsubsection{\texttt{br}: branch}
\label{sec:org71bb9a3}
Format: \texttt{0x3E <offset>} (5 bytes)

Does not modify the stack.

Skips \texttt{offset} bytes starting from after the current instruction. That
is, \texttt{br 0} is a no-op.

Note that \texttt{offset} can be negative.

\subsubsection{\texttt{jmp}: jump}
\label{sec:orgeb956da}
Format: \texttt{0x3F <address>} (5 bytes)

Jumps to \texttt{address}. Note: the current environment and stack are
unchanged.

\subsubsection{\texttt{call}: call function}
\label{sec:orgc60978b}
Format: \texttt{0x40 <numargs: u8>} (2 bytes)

Stack before: \ldots{}, \texttt{<f>: boxed (function)}, \texttt{<a1>}, \texttt{<a2>}, \ldots{},
\texttt{<aN>} \(\blacksquare\)

Stack after: \ldots{}, \texttt{<r>} \(\blacksquare\)

Pops the arguments to be passed to the function off the stack in reverse
order, followed by the function to be called. That is, pop the last
argument, followed by the second last, and so on, until the first
argument, then the function.

Calls the function, and then pushes the return value of the function
onto the stack.

The type of arguments to be popped, and the type of the return value, is
determined by the function header as referred to by the function object.

Behaviour is undefined if arguments on the stack are of different types
compared to the types specified in the function header.

A fault occurs if the number of arguments specified in \texttt{<numargs>} is
different from the number of arguments expected by the function as
specified in the function header.

\subsubsection{\texttt{call.t}: tail call function}
\label{sec:org026b562}
Format: \texttt{0x41 <numargs: u8>} (2 bytes)

Stack before: \ldots{}, \texttt{<f>: boxed (function)}, \texttt{<a1>}, \texttt{<a2>}, \ldots{},
\texttt{<aN>} \(\blacksquare\)

Stack after: (the stack is destroyed)

Pops the arguments to be passed to the function off the stack in reverse
order, followed by the function to be called. That is, pop the last
argument, followed by the second last, and so on, until the first
argument, then the function.

Calls the function. The return value of the callee function will become
the return value of the current function, and execution returns to the
caller of the current function.

The type of arguments to be popped, and the type of the return value, is
determined by the function header as referred to by the function object.

Behaviour is undefined if arguments on the stack are of different types
compared to the types specified in the function header.

A fault occurs if the number of arguments specified in \texttt{<numargs>} is
different from the number of arguments expected by the function as
specified in the function header.

\subsubsection{\texttt{call.p}: call primitive function}
\label{sec:org99bb90f}
Format: \texttt{0x42 <id: u8> <numargs: u8>} (3 bytes)

Stack before: \ldots{}, \texttt{<a1>}, \texttt{<a2>}, \ldots{}, \texttt{<aN>} \(\blacksquare\)

Stack after: \ldots{}, \texttt{<r>} \(\blacksquare\)

Pops the arguments to be passed to the function off the stack in reverse
order. That is, pop the last argument, followed by the second last, and
so on, until the first argument.

Calls the primitive function with the given \texttt{id}, and then pushes the
return value of the function onto the stack.

The types of the arguments are as specified in the list of primitive
functions.

Behaviour is undefined if arguments on the stack are of different types
compared to the specified types.

\subsubsection{\texttt{call.t.p}: tail call primitive function}
\label{sec:orgd22c74d}
Format: \texttt{0x43 <id: u8> <numargs: u8>} (3 bytes)

Stack before: \ldots{}, \texttt{<a1>}, \texttt{<a2>}, \ldots{}, \texttt{<aN>} \(\blacksquare\)

Stack after: (the stack is destroyed)

Pops the arguments to be passed to the function off the stack in reverse
order. That is, pop the last argument, followed by the second last, and
so on, until the first argument.

Calls the primitive function with the given \texttt{id}. The return value of
the primitive function will become the return value of the current
function, and execution returns to the caller of the current function.

The types of the arguments are as specified in the list of primitive
functions.

Behaviour is undefined if arguments on the stack are of different types
compared to the specified types.

\subsubsection{\texttt{call.v}: call VM-internal function/native function}
\label{sec:org008ae6e}
Format: \texttt{0x44 <id: u8> <numargs: u8>} (3 bytes)

Stack before: \ldots{}, \texttt{<a1>}, \texttt{<a2>}, \ldots{}, \texttt{<aN>} \(\blacksquare\)

Stack after: \ldots{}, \texttt{<r>} \(\blacksquare\)

Pops the arguments to be passed to the function off the stack in reverse
order. That is, pop the last argument, followed by the second last, and
so on, until the first argument.

Calls the VM-internal function with the given \texttt{id}, and then pushes the
return value of the function onto the stack.

VM-internal functions are specified by the VM implementation.

Behaviour is undefined if arguments on the stack are of different types
compared to the specified types.

\subsubsection{\texttt{call.t.v}: tail call VM-internal function/native function}
\label{sec:org22cfce9}
Format: \texttt{0x45 <id: u8> <numargs: u8>} (3 bytes)

Stack before: \ldots{}, \texttt{<a1>}, \texttt{<a2>}, \ldots{}, \texttt{<aN>} \(\blacksquare\)

Stack after: (the stack is destroyed)

Pops the arguments to be passed to the function off the stack in reverse
order. That is, pop the last argument, followed by the second last, and
so on, until the first argument.

Calls the VM-internal function with the given \texttt{id}. The return value of
the VM-internal function will become the return value of the current
function, and execution returns to the caller of the current function.

VM-internal functions are specified by the VM implementation.

Behaviour is undefined if arguments on the stack are of different types
compared to the specified types.

\subsubsection{\texttt{ret.g}: return boxed value}
\label{sec:org5c86ae1}
Format: \texttt{0x46}

Stack before: \ldots{}, \texttt{<retval>: boxed} \(\blacksquare\)

Stack after: (the stack is destroyed)

Pops \texttt{retval} off the stack. Makes \texttt{retval} the return value of the
current function, and returns execution to the caller of the current
function.

A fault occurs if the current function's signature indicates that the
type of the return value is not a boxed value.

Behaviour is undefined if \texttt{retval} is not a boxed value.

\subsubsection{\texttt{ret.f}: return number}
\label{sec:org24d7891}
Format: \texttt{0x47}

Stack before: \ldots{}, \texttt{<retval>: number} \(\blacksquare\)

Stack after: (the stack is destroyed)

Pops \texttt{retval} off the stack. Makes \texttt{retval} the return value of the
current function, and returns execution to the caller of the current
function.

A fault occurs if the current function's signature indicates that the
type of the return value is not a number.

Behaviour is undefined if \texttt{retval} is not a number.

\subsubsection{\texttt{ret.b}: return boolean}
\label{sec:org489d37c}
Format: \texttt{0x48}

Stack before: \ldots{}, \texttt{<retval>: boolean} \(\blacksquare\)

Stack after: (the stack is destroyed)

Pops \texttt{retval} off the stack. Makes \texttt{retval} the return value of the
current function, and returns execution to the caller of the current
function.

A fault occurs if the current function's signature indicates that the
type of the return value is not a boolean.

Behaviour is undefined if \texttt{retval} is not a boolean.

\subsubsection{\texttt{ret.u}: return undefined}
\label{sec:org4462eb4}
Format: \texttt{0x49}

Stack before: \ldots{} \(\blacksquare\)

Stack after: (the stack is destroyed)

Makes \texttt{undefined} the return value of the current function, and returns
execution to the caller of the current function.

A fault occurs if the current function's signature indicates that the
type of the return value is not undefined.

\subsubsection{\texttt{ret.n}: return null}
\label{sec:org02d0940}
Format: \texttt{0x4A}

Stack before: \ldots{} \(\blacksquare\)

Stack after: (the stack is destroyed)

Makes \texttt{null} the return value of the current function, and returns
execution to the caller of the current function.

A fault occurs if the current function's signature indicates that the
type of the return value is not undefined.

\subsubsection{\texttt{dup}: duplicate top of stack}
\label{sec:orgde7ff7a}
Format: \texttt{0x4B}

Stack before: \ldots{} \texttt{<value>} \(\blacksquare\)

Stack after: \ldots{} \texttt{<value>}, \texttt{<value>} \(\blacksquare\)

Pushes a copy of the topmost value on the stack onto the stack.

A fault occurs if the stack is full, or the stack is empty.

\subsubsection{\texttt{newenv}: create new environment}
\label{sec:orgb7d0128}
Format: \texttt{0x4C <size: u8>} (2 bytes)

Does not modify the stack.

Creates a new environment whose parent is the current environment and
with \texttt{size} entries. Sets the current environment to the new
environment.

\subsubsection{\texttt{popenv}: pop environment}
\label{sec:org4cc850b}
Format: \texttt{0x4D}

Does not modify the stack.

Sets the current environment to be the parent of the current
environment.

\subsection{Primitive functions}
\label{sec:org22a9c67}
The functions below correspond to the functions defined in the
\href{https://sicp.comp.nus.edu.sg/source/source\_4/}{Source language
documentation}.

The behaviours of \texttt{display} and \texttt{error} are implementation-defined.

\begin{itemize}
\item 0x00 accumulate

\item 0x01 append

\item 0x02 array\textsubscript{length}

\item 0x03 build\textsubscript{list}

\item 0x04 build\textsubscript{stream}

\item 0x05 display

\item 0x06 draw\textsubscript{data}

\item 0x07 enum\textsubscript{list}

\item 0x08 enum\textsubscript{stream}

\item 0x09 equal

\item 0x0a error

\item 0x0b eval\textsubscript{stream}

\item 0x0c filter

\item 0x0d for\textsubscript{each}

\item 0x0e head

\item 0x0f integers\textsubscript{from}

\item 0x10 is\textsubscript{array}

\item 0x11 is\textsubscript{boolean}

\item 0x12 is\textsubscript{function}

\item 0x13 is\textsubscript{list}

\item 0x14 is\textsubscript{null}

\item 0x15 is\textsubscript{number}

\item 0x16 is\textsubscript{pair}

\item 0x17 is\textsubscript{stream}

\item 0x18 is\textsubscript{string}

\item 0x19 is\textsubscript{undefined}

\item 0x1a length

\item 0x1b list

\item 0x1c list\textsubscript{ref}

\item 0x1d list\textsubscript{to}\textsubscript{stream}

\item 0x1e list\textsubscript{to}\textsubscript{string}

\item 0x1f map

\item 0x20 math\textsubscript{abs}

\item 0x21 math\textsubscript{acos}

\item 0x22 math\textsubscript{acosh}

\item 0x23 math\textsubscript{asin}

\item 0x24 math\textsubscript{asinh}

\item 0x25 math\textsubscript{atan}

\item 0x26 math\textsubscript{atan2}

\item 0x27 math\textsubscript{atanh}

\item 0x28 math\textsubscript{cbrt}

\item 0x29 math\textsubscript{ceil}

\item 0x2a math\textsubscript{clz32}

\item 0x2b math\textsubscript{cos}

\item 0x2c math\textsubscript{cosh}

\item 0x2d math\textsubscript{exp}

\item 0x2e math\textsubscript{expm1}

\item 0x2f math\textsubscript{floor}

\item 0x30 math\textsubscript{fround}

\item 0x31 math\textsubscript{hypot}

\item 0x32 math\textsubscript{imul}

\item 0x33 math\textsubscript{log}

\item 0x34 math\textsubscript{log1p}

\item 0x35 math\textsubscript{log2}

\item 0x36 math\textsubscript{log10}

\item 0x37 math\textsubscript{max}

\item 0x38 math\textsubscript{min}

\item 0x39 math\textsubscript{pow}

\item 0x3a math\textsubscript{random}

\item 0x3b math\textsubscript{round}

\item 0x3c math\textsubscript{sign}

\item 0x3d math\textsubscript{sin}

\item 0x3e math\textsubscript{sinh}

\item 0x3f math\textsubscript{sqrt}

\item 0x40 math\textsubscript{tan}

\item 0x41 math\textsubscript{tanh}

\item 0x42 math\textsubscript{trunc}

\item 0x43 member

\item 0x44 pair

\item 0x45 parse\textsubscript{int}

\item 0x46 remove

\item 0x47 remove\textsubscript{all}

\item 0x48 reverse

\item 0x49 runtime

\item 0x4a set\textsubscript{head}

\item 0x4b set\textsubscript{tail}

\item 0x4c stream

\item 0x4d stream\textsubscript{append}

\item 0x4e stream\textsubscript{filter}

\item 0x4f stream\textsubscript{for}\textsubscript{each}

\item 0x50 stream\textsubscript{length}

\item 0x51 stream\textsubscript{map}

\item 0x52 stream\textsubscript{member}

\item 0x53 stream\textsubscript{ref}

\item 0x54 stream\textsubscript{remove}

\item 0x55 stream\textsubscript{remove}\textsubscript{all}

\item 0x56 stream\textsubscript{reverse}

\item 0x57 stream\textsubscript{tail}

\item 0x58 stream\textsubscript{to}\textsubscript{list}

\item 0x59 tail

\item 0x60 stringify
\end{itemize}

\subsection{Machine-parseable instruction set}
\label{sec:org5fffbe9}
A JSON dump of the instruction set follows.

\begin{verbatim}
[[0, "nop", "no-op"], [1, "ldc.i", "load constant integer"], [2, "lgc.i", "load boxed constant integer"], [3, "ldc.f32", "load constant number (single-precision)"], [4, "lgc.f32", "load boxed constant number (single-precision)"], [5, "ldc.f64", "load constant number (double-precision)"], [6, "lgc.f64", "load boxed constant number (double-precision)"], [7, "ldc.b.0", "load constant false"], [8, "ldc.b.1", "load constant true"], [9, "lgc.b.0", "load boxed constant false"], [10, "lgc.b.1", "load boxed constant true"], [11, "lgc.u", "load boxed constant undefined"], [12, "lgc.n", "load boxed constant null"], [13, "lgc.s", "load constant string"], [14, "pop.g", "pop boxed value from stack"], [15, "pop.b", "pop boolean from stack"], [16, "pop.f", "pop number from stack"], [17, "add.g", "add boxed values"], [18, "add.f", "add numbers"], [19, "sub.g", "subtract boxed values"], [20, "sub.f", "subtract numbers"], [21, "mul.g", "multiply boxed values"], [22, "mul.f", "multiply numbers"], [23, "div.g", "divide boxed values"], [24, "div.f", "divide numbers"], [25, "mod.g", "modulo boxed values"], [26, "mod.f", "modulo numbers"], [27, "not.g", "negate boxed value"], [28, "not.b", "negate boolean"], [29, "lt.g", "less than, boxed operands"], [30, "lt.f", "less than, number operands"], [31, "gt.g", "greater than, boxed operands"], [32, "gt.f", "greater than, number operands"], [33, "le.g", "less than or equal to, boxed operands"], [34, "le.f", "less than or equal to, number operands"], [35, "ge.g", "greater than or equal to, boxed operands"], [36, "ge.f", "greater than or equal to, number operands"], [37, "eq.g", "equal, boxed operands"], [38, "eq.f", "equal, number operands"], [39, "eq.b", "equal, boolean operands"], [40, "new.c", "create function"], [41, "new.a", "create array"], [42, "ldl.g", "load boxed value from current environment"], [43, "ldl.f", "load number from current environment"], [44, "ldl.b", "load boolean from current environment"], [45, "stl.g", "store boxed value into current environment"], [46, "stl.b", "store boolean into current environment"], [47, "stl.f", "store number into current environment"], [48, "ldp.g", "load boxed value from (parent) environment"], [49, "ldp.f", "load number from (parent) environment"], [50, "ldp.b", "load boolean from (parent) environment"], [51, "stp.g", "store boxed value into (parent) environment"], [52, "stp.b", "store boolean into (parent) environment"], [53, "stp.f", "store number into (parent) environment"], [54, "lda.g", "load boxed value from array"], [55, "lda.b", "load boolean from array"], [56, "lda.f", "load number from array"], [57, "sta.g", "store boxed value into array"], [58, "sta.b", "store boolean into array"], [59, "sta.f", "store number into array"], [60, "br.t", "branch if true"], [61, "br.f", "branch if false"], [62, "br", "branch"], [63, "jmp", "jump"], [64, "call", "call function"], [65, "call.t", "tail call function"], [66, "call.p", "call primitive function"], [67, "call.t.p", "tail call primitive function"], [68, "call.v", "call VM-internal function/native function"], [69, "call.t.v", "tail call VM-internal function/native function"], [70, "ret.g", "return boxed value"], [71, "ret.f", "return number"], [72, "ret.b", "return boolean"], [73, "ret.u", "return undefined"], [74, "ret.n", "return null"], [75, "dup", "duplicate top of stack"]]
\end{verbatim}

\label{org8222156}

\label{orgc2bb649}
Last updated 2020-03-23 14:15:24 +0800
\end{document}